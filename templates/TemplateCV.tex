% FortySecondsCV LaTeX template
% Copyright © 2019 René Wirnata <rene.wirnata@pandascience.net>
% Licensed under the 3-Clause BSD License. See LICENSE file for details.
%
% Attributions
% ------------
% * fortysecondscv is based on the twentysecondcv class by Carmine Spagnuolo
%   (cspagnuolo@unisa.it), released under the MIT license and available under
%   https://github.com/spagnuolocarmine/TwentySecondsCurriculumVitae-LaTex
% * further attributions are indicated immediately before corresponding code


%-------------------------------------------------------------------------------
%                             ADDITIONAL PACKAGES
%-------------------------------------------------------------------------------
\documentclass[
  a4paper,
%  showframes,
%   maincolor=cvgreen,
%   sectioncolor=red,
%   subsectioncolor=orange
%   sidebarwidth=0.4\paperwidth,
%   topbottommargin=0.03\paperheight,
%   leftrightmargin=20pt
]{fortysecondscv}

% improve word spacing and hyphenation
\usepackage{microtype}
\usepackage{ragged2e}

% take care of proper font encoding
\ifxetex
	\usepackage{fontspec}
	\defaultfontfeatures{Ligatures=TeX}
% \newfontfamily\headingfont[Path = fonts/]{segoeuib.ttf} % local font
\else
	\usepackage[utf8x]{inputenc}
	\usepackage[T1]{fontenc}
% \usepackage[sfdefault]{noto} % use noto google font
\fi

\usepackage{upgreek}
% enable mathematical syntax for some symbols like \varnothing
\usepackage{amssymb}

% bubble diagram configuration
\usepackage{smartdiagram}
\smartdiagramset{
  % defaut font size is \large, so adjust to harmonize with sidebar layout
  bubble center node font = \footnotesize,
  bubble node font = \footnotesize,
  % default: 4cm/2.5cm; make minimum diameter relative to sidebar size
  bubble center node size = 0.4\sidebartextwidth,
  bubble node size = 0.25\sidebartextwidth,
  distance center/other bubbles = 1.5em,
  % set center bubble color
  bubble center node color = maincolor!70,
  % define the list of colors usable in the diagram
  set color list = {maincolor!10, maincolor!40,
  maincolor!20, maincolor!60, maincolor!35},
  % sets the opacity at which the bubbles are shown
  bubble fill opacity = 0.8,
}


%-------------------------------------------------------------------------------
%                            PERSONAL INFORMATION
%-------------------------------------------------------------------------------
% profile picture
\cvprofilepic{pics/matrixise.jpg}
% your name
\cvname{[[config.Author.name ]]}
% job title/career
\cvjobtitle{[[ config.Author.title ]]}
% date of birth
\cvbirthday{15 September 1980}
% short address/location, use \newline if more than 1 line is required
\cvaddress{\raggedright 6001 Marcinelle\newline Belgium}
% phone number
\cvphone{[[ config.Author.phone ]]}
% personal website
\cvgithub{[[config.Params.github]]}
\cvsite{[[config.baseurl]]}
\cvsitepro{[[config.Company.site]]}
\cvtwitter{[[config.Params.twitter]]}
% email address
\cvmail{[[config.Company.email]]}
% pgp key
%\cvkey{4096R/FF00FF00}{0xAABBCCDDFF00FF00}
% add additional information
% \newcommand{\additional}{some more?}


%-------------------------------------------------------------------------------
%                              SIDEBAR 1st PAGE
%-------------------------------------------------------------------------------
% overwrite default icons and order of personal information
 \renewcommand{\personaltable}{
 	\begin{personal}[0.8em]
% 		\circleicon{\faKey}      & \cvkey  \\
		\circleicon{\faAt}       & \cvmail \\
        \circleicon{\faTwitter}    & \cvtwitter \\
		\circleicon{\faGithub}    & \cvgithub \\
        \circleicon{\faGlobe} & \cvsite \\
        \circleicon{\faGlobe} & \cvsitepro \\
		\circleicon{\faPhone}    & \cvphone \\
		\circleicon{\faEnvelope} & \cvaddress \\\
 		% add another line
% 		\circleicon{\faQuestion} & \additional
 	\end{personal}
 }

% add more profile sections to sidebar on first page
\addtofrontsidebar{
	% include gosquare national flags from https://github.com/gosquared/flags;
	% naming according to ISO 3166-1 alpha-2 country codes
	\profilesection{About Me}
	\aboutme{
        Python enthusiast, Python Core Developer, Best Practices Automation and
        Distributed Systems are my favourite topics. Founder of Mgx.io SRL.
        Former Core Developer of Odoo}
    \graphicspath{{pics/flags/}}
	\profilesection{Prog. Languages}
	\pointskill{\flag{python.png}}{Python}{5}
	\pointskill{\flag{go.png}}{Go}{3}
	\pointskill{\flag{c++.png}}{C/C++}{3}
	\pointskill{\flag{erlang.png}}{Erlang}{2}

	\profilesection{Skills}
        \pointskill{\flag{python.png}}{Python}{5}
			\skill[1.8em]{\faLeaf}{Core Developer}
            \skill[1.8em]{\faLeaf}{CPython Extensions}
			\skill[1.8em]{\faServer}{Python packaging}
            \skill[1.8em]{\faServer}{...long list...}
}


%-------------------------------------------------------------------------------
%                              SIDEBAR 2nd PAGE
%-------------------------------------------------------------------------------
\addtobacksidebar{
	\profilesection{Skills}
		\pointskill{\faCodeFork}{DevOps}{4}
			\skill[1.8em]{\faServer}{Github, GitLab, GitLab-CI, Travis-CI, Github Actions}
            \skill[1.8em]{\faServer}{Docker, Nomad, Traefik}
            \skill[1.8em]{\faServer}{Ansible}
		\pointskill{\faCodeFork}{Software Development}{5}
 			\skill[1.8em]{\faCodeFork}{Object-oriented programming}
			\skill[1.8em]{\faCodeFork}{Unit testing, BDD}
			\skill[1.8em]{\faCodeFork}{Documentation with sphinx}
 			\skill[1.8em]{\faDatabase}{PostgreSQL, SQLite3}
 			\skill[1.8em]{\faDatabase}{Redis, RabbitMQ}
 			\skill[1.8em]{\faDatabase}{Unix environments}
 		    \skill[1.8em]{\faServer}{Best Practices}
	\profilesection{More Skills}
		\pointskill{\faConnectdevelop}{Web API}{4}
		    \skill[1.5em]{\faServer}{Django REST Framework}
            \skill[1.5em]{\faServer}{Flask, WerkZeug, Golang}
		    \skill[1.5em]{\faServer}{aiohttp/starlette/fastapi}
		    \skill[1.5em]{\faServer}{REST, GraphQL, XML-RPC}
		\pointskill{\faTv}{Web development}{3}
			\skill[1.8em]{\faServer}{Django, Flask}
			\skill[1.8em]{\faTv}{VueJS, Javascript}
		\pointskill{\faMicrophone}{Communication}{5}
			\skill[1.8em]{\faUsers}{Presenting to conferences}
    \profilesection{Memberships}
        \skill{}{Python Software Foundation}
        \skill{}{Python Ireland}
        \skill{}{EuroPython Society}
        \skill{}{Association Francophone de Python}
}



	%% High throughput data analysis
%% Archer course

	%% things I've experimented with
	% HMM's
	% Dynamic time warp algorithm
	% k means
	% Pure python ffw neural net
		% for clustering time series
	% use fa-braille for neural networks
	% built a pure python neural netowrk implementation to estimate ODE solution
%	\profilesection{Modelling Skills}
%
%	\profilesection{Transferable Skills}


%	\profilesection{Diagrams}
%	\chartlabel{Bubble Diagram}
%	\begin{figure}\centering
%		\smartdiagram[bubble diagram]{
%			\textcolor{white}{\textbf{Being a}} \\
%			\textcolor{white}{\textbf{Panda}}, % center bubble
%			\textcolor{black!90}{Eating},
%			\textcolor{black!90}{Sleeping},
%			\textcolor{black!90}{Rolling},
%			\textcolor{black!90}{Playing},
%			\textcolor{black!90}{Chilling}
%		}
%	\end{figure}
%
%	\chartlabel{Wheel Chart}
%
%	\wheelchart{4em}{2em}{
%  	20/3em/maincolor!50/Chill,
%  	15/3em/maincolor!15/Play,
%  	30/4em/maincolor!40/Sleep,
%  	20/3em/maincolor!20/Eat
%	}
%
%	\profilesection{Barskills}
%	\barskill{\faSkyatlas}{Wearing asian rice hats}{60}
%	\barskill{\faImage}{Playing Chess}{30}
%	\barskill{\faMusic}{Playing the bamboo flute}{50}
%
%	\profilesection{Memberships}
%	\begin{memberships}
%		\membership{pics/logo.png}{PandaScience.net}
%		\membership{pics/logo.png}{Here's some longer text spanning over more than
%			only one line}
%	\end{memberships}

%}


%-------------------------------------------------------------------------------
%                         TABLE ENTRIES RIGHT COLUMN
%-------------------------------------------------------------------------------
\begin{document}

\makefrontsidebar

[% import "cv-item.tex" as cv_item %]
\cvsection{Current Position}
\begin{cvtable}
[%- for position in position_loader.get_current_positions() -%]
[[- cv_item.cv_item(position) -]]
[%- endfor -%]
\end{cvtable}

\cvsection{Past Positions}
\begin{cvtable}
[% for position in position_loader.get_other_positions() %]
[[ cv_item.cv_item(position) ]]
[% endfor %]
\end{cvtable}

$% \cvsection{Current Occupation}
$% %\subsection{}
$% \begin{cvtable}
$% 	\cvitemx{Research Associate in MESI-STRAT Project -- 2018}{Newcastle University}
$% 		{A modeller in the MESI-STRAT project (www.mesi-strat.eu), an initiative
$% 		for the optimization of treatment strategies for breast cancer patients. The goal is to combine molecular
$% 		biology with mathematical modelling to predict disease progression and response to treatments. This includes abstraction of
$% 		biological processes into mathematical formalisms using expert knowledge and biochemistry theory, training models
$% 		on dynamic perturbation data, testing on unseen validation data and selecting between multiple model hypotheses.
$% 		This work involves frequent use of Newcastle University's HPC supercomputer for intense computation.
$% 		}
$% \end{cvtable}

% \cvsection{\Large PhD in Computational Systems Biology -- 2014 -- 2018}
% \begin{cvtable}
% 		%% A focus on interdisaplinary
% 		%% attended and presented at conferences
% 	\cvitemx{Skin Ageing Research Project}{Newcastle University}
% 		{An computational investigation into the changes in cell signalling which lead to the skin ageing phenotype,
% 		 specifically in reduced collagen content with age. Designed and analysed microarray and
% 		high-throughput qPCR experiments to study changes in TGF-$\upbeta{}$ signalling network with age.
% 		This data was then used to inform a mechanistic model that predicts the relative quantity
% 		of collagen produced in young and old human dermal fibroblasts, given the relative
% 		concentration of other network components.
% 		}
% 	\cvitemx{Programming and Software Development}{Newcastle University}
% 	{Programming abilities matured through development of two Python packages, 1) 'PyCoTools`, a package
% 	for ODE modelling and 2) 'pytseries` a package for analysis of time series data.
% 	PyCoTools is essentially a third party API for COPASI, an application for modelling biological systems.
% 	Facilities are included for building and numerically solving deterministic or stochastic models, parameter estimation, model selection,
% 	identifiability analysis, sensitivity analysis and exploratory data analysis
% 	on parameter estimation data. Pytseries provides convenient classes
% 	for manipulation of time series objects, an implementation of the
% 	dynamic time warping (DTW) algorithm and time series clustering using the k-means with the DTW distance. }
% 	\cvitemx{Interdisciplinary Research}{Newcastle University}
% 		{Well versed in the challenges of interdisciplinary research.
% 		Experience to date has provided a solid platform from which to develop
% 		proficiency in computer science, mathematics, molecular biology and
% 		general research skills.}
% 	\cvitemx{Collaboration with Industry Partners}{Newcastle University}
% 		{Project was funded by P\&G, Cincinnati. This necessitated
% 		regular monthly progress reports via telecoms and
% 		detailed presentations once a year to the board of directors.
% 		}
% \end{cvtable}

% \section{MSc in Computational Systems Biology -- 2014}
% \begin{cvtable}
% 	\cvitemx{Modules}{Distinction -- Newcastle University}
% 		{Modelling Cellular Systems; Introductory Programming for Biologists;
% 		Research Skills for Bioinformatics; Numeric Skills (Statistics and Mathematics);
% 		Stochastic Systems Biology; Systems Biology; Computing Environments for Bioinformatics;
% 		Advanced Object-Oriented Programming; Research Project: A systems biology investigation into
% 		NF-$\upkappa$B signalling during oxidative stress.
% 		}
% \end{cvtable}

\cvsection{Talks}
\begin{itemize}
[% for talk in talk_loader.talks %]
\item{[[talk['title']]]}{}{ }
[% endfor %]
\end{itemize}

\newpage

\makebacksidebar

\cvsection{Conferences}
\begin{cvtable}
    \cvitemx{EuroPython 2015, 2016, 2017, 2018}{} {Speaker, Co-Organiser, Volunteer}
    \cvitemx{PyCon Canada 2014, 2015, 2016, 2017}{} {Speaker}
    \cvitemx{PyCon France 2010, 2012, 2013, 2014, 2015, 2017, 2018}{} {Volunteer, Speaker}
    \cvitemx{PyCon Germany 2018, 2019}{} {Speaker}
    \cvitemx{PyCon Ireland 2014, 2015, 2016, 2017, 2018, 2019}{} {Speaker}
    \cvitemx{PyCon Italy 2018}{} {Speaker}
    \cvitemx{PyCon Slovakia 2018}{} {Speaker, Keynoter}
    \cvitemx{PyCon UK 2014, 2015, 2016, 2018}{} {Speaker}
    \cvitemx{PyCon US 2014, 2015, 2016, 2017, 2018, 2019}{} {Volunteer, Helper for Tutorials, Language Summit}
    \cvitemx{PyCon Ukraine 2018}{}{Keynoter}
    \cvitemx{PythonFOSDEM 2013, 2014, 2015, 2016, 2017}{} {Co-Organiser, Speaker}
\end{cvtable}
% \section{BSc in Biochemistry -- 2009 -- 2013}
% \begin{cvtable}
% 	\cvitemx{Final Year Modules}{2:1 -- University of Surrey}
% 		{Biochemistry, Immunology, Systems Biology, Pharmacology,
% 		Neuroscience, Cancer Therapeutics and Disease and a
% 		Research Project}
% 	\cvitemx{Erasmus Placement Year -- 2011 -- 2012}{University of Cagliari, Italy}
% 		{During this year I spent time shadowing an experimental scientist in a research environment. The project revolved
% 		around investigating the influence of caffeine on MDMA-induced neuroinflammation in mice. I was involved in
% 		performing and quantifying experimental data and interpreting the results. }
% \end{cvtable}
%\subsection{Year in industry}

%%% \begin{sidebar}
%%%     \nameandjob
%%%     \profilesection{Interests}
%%%         \skill{}{Programming / Oenology / Nature}
%%%     \profilesection{Sports}
%%%         \skill{}{Running, Walking, Cycling}
%%%         \skill{}{Karate Shotokan JKA Shodan}
%%% 	\graphicspath{{pics/flags/}}
%%% 	\profilesection{Languages}
%%%     \pointskill{\flag{france.png}}{French}{5}
%%%     \pointskill{\flag{english.jpg}}{English}{3}
%%% 	\pointskill{\flag{italian.png}}{Italian}{2}
%%% 	\pointskill{\flag{dutch.jpg}}{Dutch}{1}
%%% \end{sidebar}
\cvsection{Python}
\begin{cvtable}
    \cvitemx{Python Ireland}{Co-Director}{Promote Python in Ireland}
    \cvitemx{EuroPython Society -- 2019}{Board Member}{Promoted as Board Member of EuroPython Society -- Co-Organizer of EuroPython 2020}
    \cvitemx{CPython Core Developer -- 2019}{Commit Rights}{Promoted as Python Core Dev since April 2019}
	\cvitemx{CPython Contributor -- 2014}{Bug Triager}{Several contributions to CPython 3.x, devguide, planet, other projects}
    \cvitemx{Member of EuroPython Society -- 2014}{}{Member of the workgroups
    for the organisation of EuroPython}
    \cvitemx{Python Software Foundation -- 2016}{Community Service
    Award}{Received a Community Service Award during PyCon US 2017 in Portland for my contributions
    to the Python language and for the creation of PythonFOSDEM (\textasciitilde800 people)}
    \cvitemx{Python Software Foundation -- 2015}{Members of Workgroups}{
        Member of the \#fellow and \#marketing workgroups of the Python Software
        Foundation}
	\cvitemx{Python Software Foundation -- 2013}{Fellow Member}{  }
\end{cvtable}

\cvsignature

\end{document}
